% generated by GAPDoc2LaTeX from XML source (Frank Luebeck)
\documentclass[a4paper,11pt]{report}

\usepackage{a4wide}
\sloppy
\pagestyle{myheadings}
\usepackage{amssymb}
\usepackage[latin1]{inputenc}
\usepackage{makeidx}
\makeindex
\usepackage{color}
\definecolor{FireBrick}{rgb}{0.5812,0.0074,0.0083}
\definecolor{RoyalBlue}{rgb}{0.0236,0.0894,0.6179}
\definecolor{RoyalGreen}{rgb}{0.0236,0.6179,0.0894}
\definecolor{RoyalRed}{rgb}{0.6179,0.0236,0.0894}
\definecolor{LightBlue}{rgb}{0.8544,0.9511,1.0000}
\definecolor{Black}{rgb}{0.0,0.0,0.0}

\definecolor{linkColor}{rgb}{0.0,0.0,0.554}
\definecolor{citeColor}{rgb}{0.0,0.0,0.554}
\definecolor{fileColor}{rgb}{0.0,0.0,0.554}
\definecolor{urlColor}{rgb}{0.0,0.0,0.554}
\definecolor{promptColor}{rgb}{0.0,0.0,0.589}
\definecolor{brkpromptColor}{rgb}{0.589,0.0,0.0}
\definecolor{gapinputColor}{rgb}{0.589,0.0,0.0}
\definecolor{gapoutputColor}{rgb}{0.0,0.0,0.0}

%%  for a long time these were red and blue by default,
%%  now black, but keep variables to overwrite
\definecolor{FuncColor}{rgb}{0.0,0.0,0.0}
%% strange name because of pdflatex bug:
\definecolor{Chapter }{rgb}{0.0,0.0,0.0}
\definecolor{DarkOlive}{rgb}{0.1047,0.2412,0.0064}


\usepackage{fancyvrb}

\usepackage{mathptmx,helvet}
\usepackage[T1]{fontenc}
\usepackage{textcomp}


\usepackage[
            pdftex=true,
            bookmarks=true,        
            a4paper=true,
            pdftitle={Written with GAPDoc},
            pdfcreator={LaTeX with hyperref package / GAPDoc},
            colorlinks=true,
            backref=page,
            breaklinks=true,
            linkcolor=linkColor,
            citecolor=citeColor,
            filecolor=fileColor,
            urlcolor=urlColor,
            pdfpagemode={UseNone}, 
           ]{hyperref}

\newcommand{\maintitlesize}{\fontsize{50}{55}\selectfont}

% write page numbers to a .pnr log file for online help
\newwrite\pagenrlog
\immediate\openout\pagenrlog =\jobname.pnr
\immediate\write\pagenrlog{PAGENRS := [}
\newcommand{\logpage}[1]{\protect\write\pagenrlog{#1, \thepage,}}
%% were never documented, give conflicts with some additional packages

\newcommand{\GAP}{\textsf{GAP}}

%% nicer description environments, allows long labels
\usepackage{enumitem}
\setdescription{style=nextline}

%% depth of toc
\setcounter{tocdepth}{1}





%% command for ColorPrompt style examples
\newcommand{\gapprompt}[1]{\color{promptColor}{\bfseries #1}}
\newcommand{\gapbrkprompt}[1]{\color{brkpromptColor}{\bfseries #1}}
\newcommand{\gapinput}[1]{\color{gapinputColor}{#1}}


\begin{document}

\logpage{[ 0, 0, 0 ]}
\begin{titlepage}
\mbox{}\vfill

\begin{center}{\maintitlesize \textbf{\textsf{hypergraphs}\mbox{}}}\\
\vfill

\hypersetup{pdftitle=\textsf{hypergraphs}}
\markright{\scriptsize \mbox{}\hfill \textsf{hypergraphs} \hfill\mbox{}}
{\Huge \textbf{A GAP package to work with hypergraphs\mbox{}}}\\
\vfill

{\Huge Version 0.1\mbox{}}\\[1cm]
{16 February 2016\mbox{}}\\[1cm]
\mbox{}\\[2cm]
{\Large \textbf{Bert{\a'\i}n Hern{\a'a}ndez-Trejo  \mbox{}}}\\
{\Large \textbf{ Rafael Villarroel-Flores   \mbox{}}}\\
{\Large \textbf{Citlalli Zamora-Mej{\a'\i}a  \mbox{}}}\\
\hypersetup{pdfauthor=Bert{\a'\i}n Hern{\a'a}ndez-Trejo  ;  Rafael Villarroel-Flores   ; Citlalli Zamora-Mej{\a'\i}a  }
\end{center}\vfill

\mbox{}\\
{\mbox{}\\
\small \noindent \textbf{Bert{\a'\i}n Hern{\a'a}ndez-Trejo  }  Email: \href{mailto://bertin13@gmail.com} {\texttt{bertin13@gmail.com}}}\\
{\mbox{}\\
\small \noindent \textbf{ Rafael Villarroel-Flores   }  Email: \href{mailto://rvf0068@gmail.com} {\texttt{rvf0068@gmail.com}}\\
  Homepage: \href{http://rvf0068.github.io} {\texttt{http://rvf0068.github.io}}}\\
{\mbox{}\\
\small \noindent \textbf{Citlalli Zamora-Mej{\a'\i}a  }  Email: \href{mailto://cizame@gmail.com} {\texttt{cizame@gmail.com}}}\\
\end{titlepage}

\newpage\setcounter{page}{2}
{\small 
\section*{Copyright}
\logpage{[ 0, 0, 1 ]}
 {\copyright} 2016 by Bert{\a'\i}n Hern{\a'a}ndez-Trejo, Rafael
Villarroel-Flores and Citlalli Zamora-Mej{\a'\i}a

 \textsf{hypergraphs} package is free software; you can redistribute it and/or modify it under the
terms of the \href{http://www.fsf.org/licenses/gpl.html} {GNU General Public License} as published by the Free Software Foundation; either version 2 of the License,
or (at your option) any later version. \mbox{}}\\[1cm]
\newpage

\def\contentsname{Contents\logpage{[ 0, 0, 2 ]}}

\tableofcontents
\newpage

 
\chapter{\textcolor{Chapter }{Hypergraph Objects}}\label{objects}
\logpage{[ 1, 0, 0 ]}
\hyperdef{L}{X847C8187865F15C7}{}
{
  
\section{\textcolor{Chapter }{Hypergraph}}\label{hypergraph}
\logpage{[ 1, 1, 0 ]}
\hyperdef{L}{X87773B028334B037}{}
{
  }

 }

 
\chapter{\textcolor{Chapter }{Basic Constructions}}\label{basic}
\logpage{[ 2, 0, 0 ]}
\hyperdef{L}{X8247E4C57C519F9F}{}
{
  
\section{\textcolor{Chapter }{Hypergraphs}}\label{Hypergraphs}
\logpage{[ 2, 1, 0 ]}
\hyperdef{L}{X86B1C2247FC33F84}{}
{
  

\subsection{\textcolor{Chapter }{HHypergraph (for list of vertices and edges)}}
\logpage{[ 2, 1, 1 ]}\nobreak
\hyperdef{L}{X85A3657E7C167D41}{}
{\noindent\textcolor{FuncColor}{$\triangleright$\ \ \texttt{HHypergraph({\mdseries\slshape V, Ed})\index{HHypergraph@\texttt{HHypergraph}!for list of vertices and edges}
\label{HHypergraph:for list of vertices and edges}
}\hfill{\scriptsize (method)}}\\
\noindent\textcolor{FuncColor}{$\triangleright$\ \ \texttt{HHypergraph({\mdseries\slshape Ed})\index{HHypergraph@\texttt{HHypergraph}!for only edges}
\label{HHypergraph:for only edges}
}\hfill{\scriptsize (method)}}\\


 Returns the hypergraph object, with vertices \mbox{\texttt{\mdseries\slshape V}} and hyperedges \mbox{\texttt{\mdseries\slshape Ed}}. In the second form, the hyperedges determine the set of vertices, as the
union of the hyperedges. }

 

\subsection{\textcolor{Chapter }{HCompleteHypergraph}}
\logpage{[ 2, 1, 2 ]}\nobreak
\hyperdef{L}{X7FD4832B7838D3A8}{}
{\noindent\textcolor{FuncColor}{$\triangleright$\ \ \texttt{HCompleteHypergraph({\mdseries\slshape n, r})\index{HCompleteHypergraph@\texttt{HCompleteHypergraph}}
\label{HCompleteHypergraph}
}\hfill{\scriptsize (function)}}\\


 Returns the hypergraph that has $\{1\ldots \mbox{\texttt{\mdseries\slshape n}}\}$ as set of vertices, and all \mbox{\texttt{\mdseries\slshape r}}-subsets of $\{1\ldots \mbox{\texttt{\mdseries\slshape n}}\}$ as hyperedges. }

 

\subsection{\textcolor{Chapter }{HRandomUniformHypergraph}}
\logpage{[ 2, 1, 3 ]}\nobreak
\hyperdef{L}{X7C6E5B2581C7EB44}{}
{\noindent\textcolor{FuncColor}{$\triangleright$\ \ \texttt{HRandomUniformHypergraph({\mdseries\slshape n, r, p})\index{HRandomUniformHypergraph@\texttt{HRandomUniformHypergraph}}
\label{HRandomUniformHypergraph}
}\hfill{\scriptsize (function)}}\\


 Returns a hypergraph with set of vertices given by $\{1\ldots\mbox{\texttt{\mdseries\slshape n}}\}$, and where each \mbox{\texttt{\mdseries\slshape r}}-subset of $\{1\ldots\mbox{\texttt{\mdseries\slshape n}}\}$ appears as a hyperedge with probability \mbox{\texttt{\mdseries\slshape p}}. }

 

\subsection{\textcolor{Chapter }{HRemovedEdge}}
\logpage{[ 2, 1, 4 ]}\nobreak
\hyperdef{L}{X7A52BE4C7C48347D}{}
{\noindent\textcolor{FuncColor}{$\triangleright$\ \ \texttt{HRemovedEdge({\mdseries\slshape H, e})\index{HRemovedEdge@\texttt{HRemovedEdge}}
\label{HRemovedEdge}
}\hfill{\scriptsize (function)}}\\


 Returns the graph obtained from the hypergraph \mbox{\texttt{\mdseries\slshape H}} removing its edge \mbox{\texttt{\mdseries\slshape e}}. }

 

\subsection{\textcolor{Chapter }{HRemovedVertex}}
\logpage{[ 2, 1, 5 ]}\nobreak
\hyperdef{L}{X7C3E55647FC75C20}{}
{\noindent\textcolor{FuncColor}{$\triangleright$\ \ \texttt{HRemovedVertex({\mdseries\slshape H, x})\index{HRemovedVertex@\texttt{HRemovedVertex}}
\label{HRemovedVertex}
}\hfill{\scriptsize (function)}}\\


 Returns the hypergraph obtained from the hypergraph \mbox{\texttt{\mdseries\slshape H}} by removing the vertex \mbox{\texttt{\mdseries\slshape x}} from its list of vertices and from each of its edges. It also removes edges
that become empty as a result. }

 }

 
\section{\textcolor{Chapter }{Properties}}\label{Properties}
\logpage{[ 2, 2, 0 ]}
\hyperdef{L}{X871597447BB998A1}{}
{
  

\subsection{\textcolor{Chapter }{IsUniform}}
\logpage{[ 2, 2, 1 ]}\nobreak
\hyperdef{L}{X8735FBE180797557}{}
{\noindent\textcolor{FuncColor}{$\triangleright$\ \ \texttt{IsUniform({\mdseries\slshape H})\index{IsUniform@\texttt{IsUniform}}
\label{IsUniform}
}\hfill{\scriptsize (method)}}\\


 Determines if the hypergraph \mbox{\texttt{\mdseries\slshape H}} is uniform, that is, if all edges of \mbox{\texttt{\mdseries\slshape H}} have the same cardinality $k$. If \mbox{\texttt{\mdseries\slshape H}} is uniform, then the function returns $k$, otherwise, it returns false. }

 

\subsection{\textcolor{Chapter }{IsSimple}}
\logpage{[ 2, 2, 2 ]}\nobreak
\hyperdef{L}{X7D8E63A7824037CC}{}
{\noindent\textcolor{FuncColor}{$\triangleright$\ \ \texttt{IsSimple({\mdseries\slshape H})\index{IsSimple@\texttt{IsSimple}}
\label{IsSimple}
}\hfill{\scriptsize (method)}}\\


 Determines whether the hypergraph \mbox{\texttt{\mdseries\slshape H}} is simple. (A hypergraph is simple if no edge is contained in another edge.) }

 

\subsection{\textcolor{Chapter }{IsConnected}}
\logpage{[ 2, 2, 3 ]}\nobreak
\hyperdef{L}{X877F178084D9A589}{}
{\noindent\textcolor{FuncColor}{$\triangleright$\ \ \texttt{IsConnected({\mdseries\slshape H})\index{IsConnected@\texttt{IsConnected}}
\label{IsConnected}
}\hfill{\scriptsize (method)}}\\


 Determines whether the hypergraph \mbox{\texttt{\mdseries\slshape H}} is connected. }

 }

 
\section{\textcolor{Chapter }{Parameters}}\label{Parameters}
\logpage{[ 2, 3, 0 ]}
\hyperdef{L}{X7F0090D779780884}{}
{
  

\subsection{\textcolor{Chapter }{HDistance}}
\logpage{[ 2, 3, 1 ]}\nobreak
\hyperdef{L}{X8540BDF07EA97CFB}{}
{\noindent\textcolor{FuncColor}{$\triangleright$\ \ \texttt{HDistance({\mdseries\slshape H, x, y})\index{HDistance@\texttt{HDistance}}
\label{HDistance}
}\hfill{\scriptsize (function)}}\\


 Given a hypergraph \mbox{\texttt{\mdseries\slshape H}} and two of its vertices \mbox{\texttt{\mdseries\slshape x}}, \mbox{\texttt{\mdseries\slshape y}}, this function returns the distance in \mbox{\texttt{\mdseries\slshape H}} from \mbox{\texttt{\mdseries\slshape x}} to \mbox{\texttt{\mdseries\slshape y}}. }

 

\subsection{\textcolor{Chapter }{HDiameter}}
\logpage{[ 2, 3, 2 ]}\nobreak
\hyperdef{L}{X83CBF1AD82B8602F}{}
{\noindent\textcolor{FuncColor}{$\triangleright$\ \ \texttt{HDiameter({\mdseries\slshape H})\index{HDiameter@\texttt{HDiameter}}
\label{HDiameter}
}\hfill{\scriptsize (method)}}\\


 Returns the diameter of the hypergraph \mbox{\texttt{\mdseries\slshape H}}. }

 

\subsection{\textcolor{Chapter }{HGirth}}
\logpage{[ 2, 3, 3 ]}\nobreak
\hyperdef{L}{X7E61DDA6837D9EAA}{}
{\noindent\textcolor{FuncColor}{$\triangleright$\ \ \texttt{HGirth({\mdseries\slshape H})\index{HGirth@\texttt{HGirth}}
\label{HGirth}
}\hfill{\scriptsize (method)}}\\


 Returns the girth of the hypergraph \mbox{\texttt{\mdseries\slshape H}}. }

 }

 
\section{\textcolor{Chapter }{Lists}}\label{Lists}
\logpage{[ 2, 4, 0 ]}
\hyperdef{L}{X7B256AE5780F140A}{}
{
  

\subsection{\textcolor{Chapter }{Vertices}}
\logpage{[ 2, 4, 1 ]}\nobreak
\hyperdef{L}{X79E4BB4F849AC8A1}{}
{\noindent\textcolor{FuncColor}{$\triangleright$\ \ \texttt{Vertices({\mdseries\slshape H})\index{Vertices@\texttt{Vertices}}
\label{Vertices}
}\hfill{\scriptsize (method)}}\\


 Returns the list of vertices of the hypergraph \mbox{\texttt{\mdseries\slshape H}}. }

 

\subsection{\textcolor{Chapter }{Edges}}
\logpage{[ 2, 4, 2 ]}\nobreak
\hyperdef{L}{X805DA3C47BF09BD1}{}
{\noindent\textcolor{FuncColor}{$\triangleright$\ \ \texttt{Edges({\mdseries\slshape H})\index{Edges@\texttt{Edges}}
\label{Edges}
}\hfill{\scriptsize (method)}}\\


 Returns the list of edges of the hypergraph \mbox{\texttt{\mdseries\slshape H}}. }

 

\subsection{\textcolor{Chapter }{HNeighborhood}}
\logpage{[ 2, 4, 3 ]}\nobreak
\hyperdef{L}{X8598D1FF7A9EE606}{}
{\noindent\textcolor{FuncColor}{$\triangleright$\ \ \texttt{HNeighborhood({\mdseries\slshape H, x})\index{HNeighborhood@\texttt{HNeighborhood}}
\label{HNeighborhood}
}\hfill{\scriptsize (function)}}\\


 Given a hypergraph \mbox{\texttt{\mdseries\slshape H}} and one of its vertices \mbox{\texttt{\mdseries\slshape x}}, returns the set of vertices that share an edge with \mbox{\texttt{\mdseries\slshape x}}. }

 

\subsection{\textcolor{Chapter }{HDistancesFrom}}
\logpage{[ 2, 4, 4 ]}\nobreak
\hyperdef{L}{X787895877CEB1CED}{}
{\noindent\textcolor{FuncColor}{$\triangleright$\ \ \texttt{HDistancesFrom({\mdseries\slshape H, x})\index{HDistancesFrom@\texttt{HDistancesFrom}}
\label{HDistancesFrom}
}\hfill{\scriptsize (function)}}\\


 Given a hypergraph \mbox{\texttt{\mdseries\slshape H}} and one of its vertices \mbox{\texttt{\mdseries\slshape x}}, it returns a record \mbox{\texttt{\mdseries\slshape L}}, where \mbox{\texttt{\mdseries\slshape L.u}} is equal to the distance in \mbox{\texttt{\mdseries\slshape H}} from the vertex \mbox{\texttt{\mdseries\slshape x}} to the vertex \mbox{\texttt{\mdseries\slshape u}}. }

 

\subsection{\textcolor{Chapter }{IndexOfEdges}}
\logpage{[ 2, 4, 5 ]}\nobreak
\hyperdef{L}{X78DAC8357DDDB349}{}
{\noindent\textcolor{FuncColor}{$\triangleright$\ \ \texttt{IndexOfEdges({\mdseries\slshape H})\index{IndexOfEdges@\texttt{IndexOfEdges}}
\label{IndexOfEdges}
}\hfill{\scriptsize (method)}}\\


 Given a hypergraph \mbox{\texttt{\mdseries\slshape H}}, the function returns a record \mbox{\texttt{\mdseries\slshape I}}, where $I.u$ is a list of the indices of the edges where the vertex \mbox{\texttt{\mdseries\slshape u}} appears. }

 }

 }

 
\chapter{\textcolor{Chapter }{Library of Hypergraphs}}\label{library}
\logpage{[ 3, 0, 0 ]}
\hyperdef{L}{X7D7EDEB07A16E809}{}
{
  
\section{\textcolor{Chapter }{Hypergraphs}}\label{libhypergraphs}
\logpage{[ 3, 1, 0 ]}
\hyperdef{L}{X86B1C2247FC33F84}{}
{
  

\subsection{\textcolor{Chapter }{HFano}}
\logpage{[ 3, 1, 1 ]}\nobreak
\hyperdef{L}{X7A43BC9478512221}{}
{\noindent\textcolor{FuncColor}{$\triangleright$\ \ \texttt{HFano\index{HFano@\texttt{HFano}}
\label{HFano}
}\hfill{\scriptsize (global variable)}}\\


 The Fano hypergraph. }

 

\subsection{\textcolor{Chapter }{HQuad}}
\logpage{[ 3, 1, 2 ]}\nobreak
\hyperdef{L}{X862906887B40D8C6}{}
{\noindent\textcolor{FuncColor}{$\triangleright$\ \ \texttt{HQuad\index{HQuad@\texttt{HQuad}}
\label{HQuad}
}\hfill{\scriptsize (global variable)}}\\


 The hypergraph of the smallest generalized quadrangle. }

 }

 }

 \def\indexname{Index\logpage{[ "Ind", 0, 0 ]}
\hyperdef{L}{X83A0356F839C696F}{}
}

\cleardoublepage
\phantomsection
\addcontentsline{toc}{chapter}{Index}


\printindex

\newpage
\immediate\write\pagenrlog{["End"], \arabic{page}];}
\immediate\closeout\pagenrlog
\end{document}
